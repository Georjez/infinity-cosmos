% In this file you should put all LaTeX macros to be used
% both by the pdf version and the web version.
% This should be most of your macros.

% Theorems
\theoremstyle{theorem}
\newtheorem{theorem}{Theorem}[section]%for blueprint
\newtheorem{thm}[theorem]{Theorem}%not for blueprint
\newtheorem{lemma}[theorem]{Lemma}
\newtheorem{lem}[theorem]{Lemma}
\newtheorem{proposition}[theorem]{Proposition}
\newtheorem{prop}[theorem]{Proposition}
\newtheorem{cor}[theorem]{Corollary}
\newtheorem{conj}[theorem]{Conjecture}
\newtheorem{rmk}[theorem]{Remark}
\newtheorem{obs}[theorem]{Observation}
\newtheorem{dig}[theorem]{Digression}
\newtheorem{rec}[theorem]{Recall}
\newtheorem{war}[theorem]{Warning}

% Definitions
\theoremstyle{definition}
\newtheorem{definition}[theorem]{Definition}
\newtheorem{defn}[theorem]{Definition}
\newtheorem{ex}[theorem]{Example}
\newtheorem{nex}[theorem]{Non-Example}
\newtheorem{ntn}[theorem]{Notation}
\newtheorem{con}[theorem]{Convention}

% Number equations in sections
\makeatletter
\let\c@equation\c@theorem
\numberwithin{equation}{section}
\makeatother

\renewcommand{\theenumi}{\roman{enumi}}

%enriched (i.e. all large) categories.
\newcommand{\ec}[1]{{\mathord{\mathcal{#1}}}}
\newcommand{\cA}{\ec{A}}
\newcommand{\cC}{\ec{C}}
\newcommand{\cD}{\ec{D}}
\newcommand{\cK}{\ec{K}}
\newcommand{\cL}{\ec{L}}
\newcommand{\cV}{\ec{V}}
\newcommand{\cW}{\ec{W}}
\newcommand{\Cat}{\ec{Cat}}
\newcommand{\Kan}{\ec{Kan}}
\newcommand{\qCat}{\ec{QCat}}
\newcommand{\Set}{\ec{Set}}
\newcommand{\sSet}{\ec{sSet}}
\newcommand{\twoCat}{2\text{-}\Cat}
\newcommand{\sCat}{\sSet\text{-}\Cat}
\newcommand{\eCat}[1]{{#1}\text{-}\Cat}%for enriched categories

% Various hom notations.
\newcommand{\qop}[1]{{\mathord{\mathsf{#1}}}}
\newcommand{\Fun}{\qop{Fun}}%mapping spaces in cosmoi
\newcommand{\ho}{\qop{h}}
\newcommand{\hFun}{\qop{hFun}}%hom categories in homotopy 2-categories
\newcommand{\core}{\qop{core}}%not cosmological
\newcommand{\coreop}[1]{{{#1}^\simeq}}%for core as an operator

\newcommand{\catfour}{{\Bbbfour}}%texdoc unimath-symbols.pdf
\newcommand{\catthree}{{\Bbbthree}}%texdoc unimath-symbols.pdf
\newcommand{\cattwo}{{\Bbbtwo}}
\newcommand{\catone}{{\Bbbone}}
\newcommand{\catn}{{\mathbb{n}}}
\newcommand{\catnone}{{\catn\!+\!\catone}}
\newcommand{\catntwo}{{\catn\!+\!\cattwo}}
\newcommand{\iso}{{\BbbI}}

%arrows; cheap version of \we and \trvfib
\makeatletter
\def\makeslashed#1#2#3#4#5{#1{\mathpalette{\sla@{#2}{#3}{#4}}{#5}}}

\def\@mathlower#1#2#3{\setbox0=\hbox{$\m@th#2#3$}\lower#1\ht0\box0}
\def\mathlower#1#2{\mathpalette{\@mathlower{#1}}{#2}}
\makeatother

\newcommand\dhxrightarrow[2][]{%
  \mathrel{\ooalign{$\xrightarrow[#1\mkern4mu]{#2\mkern4mu}$\cr%
  \hidewidth$\rightarrow\mkern4mu$}}
}

% New style, simpler, inline arrows.
\let\xto=\xrightarrow
\newcommand{\fib}{\twoheadrightarrow}
\newcommand{\we}{\xrightarrow{{\smash{\mathlower{0.8}{\sim}}}}}
\newcommand{\trvfib}{\dhxrightarrow{{\smash{\mathlower{0.8}{\sim}}}}}
\newcommand{\To}{\Rightarrow}%2-cells
\newcommand{\isoto}{\xrightarrow{{\smash{\mathlower{0.8}{\cong}}}}}
\newcommand{\inc}{\hookrightarrow}

%1-cells induced from 2-cells
\newcommand{\name}[1]{{\ulcorner{#1}\urcorner}}


% Elementary operators in the theory of (stratified) simplicial sets.

\newcommand{\face}{\delta}
\newcommand{\degen}{\sigma}
\newcommand{\fbv}[1]{\{{#1}\}}


% Duals / Superscripted postfix ops

\newcommand{\op}{^{\mathord{\textup{op}}}}
\newcommand{\co}{^{\mathord{\textup{co}}}}
\newcommand{\coop}{^{\mathord{\textup{coop}}}}

% General operations on maps etc.
\newcommand{\cosk}{\textup{cosk}}
\newcommand{\cod}{\textup{cod}}
\newcommand{\dom}{\textup{dom}}
\newcommand{\ev}{\textup{ev}}
\newcommand{\id}{\textup{id}}
\newcommand{\sk}{\textup{sk}}


%###trial notation for the homotopy 2-category
\newcommand{\h}{\mathfrak{h}}

% Leibniz...
% ... for binary operators like join and product
\newcommand{\leib}[1]{\mathbin{\widehat{#1}}}
% ... for prefix operators like lim, colim and decalage.
\newcommand{\uleib}[1]{\widehat{#1}}
\newcommand{\pwr}{\pitchfork}
